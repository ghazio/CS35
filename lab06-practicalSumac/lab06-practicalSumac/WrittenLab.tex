% This LaTeX file contains your written lab questions.  You may answer these
% questions just by inserting your answer into this document.
%
% If you're unfamiliar with LaTeX, see the document LearningLaTeX.tex in this
% same directory.  It contains a brief explanation and a few snippets of LaTeX
% code to get you started; in fact, it should have everything you need to
% complete this assignment.
\documentclass{article}

\usepackage{amsmath}
\usepackage{amssymb}
\usepackage{amsthm}
\usepackage{algpseudocode}
\usepackage{algorithmicx}
\usepackage{enumerate}

\newtheorem{claim}{Claim}

\begin{document}
    \section{Inductive Proofs}

    Prove each of the following claims by induction

    \begin{claim}
      The sum of the first $n$ even numbers is $n^2+n$.  That is, $\sum\limits_{i=1}^n (2i) = n^2+n$
    \end{claim}

      %%%%%%%%%%%%%%
      %%%% TODO %%%%
      %%%%%%%%%%%%%%
      % Write your proof of the above statement here.


    \begin{claim}
      $\sum\limits_{i=1}^{n} \left(\dfrac{2}{3}\right)^i = 1 - \dfrac{1}{3^n}$
    \end{claim}

      %%%%%%%%%%%%%%
      %%%% TODO %%%%
      %%%%%%%%%%%%%%
      % Write your proof of the above statement here.


    \begin{claim}
      For every $n \geq 1$, $5^n - 1$ is divisible by $4$.  In other words, for every $n \geq 1$, there exists some integer constant $z_n$ such that $5^n - 1 = 4z_n$.  Note that each power of $5$ has a different $z$.
    \end{claim}

      %%%%%%%%%%%%%%
      %%%% TODO %%%%
      %%%%%%%%%%%%%%
      % Write your proof of the above statement here.


    \vspace{1cm}
    \section{Recursive Invariants}

    The function \texttt{minPos}, given below in pseudocode, takes as input an array $A$ of size $n$ numbers and returns the smallest \text{positive} number in the array.  If no positive numbers appear in the array, it returns zero.  (Note that zero is neither positive nor negative.)  Using induction, prove that the \texttt{minPos} function works correctly.  Clearly state your recursive invariant at the beginning of your proof.

    \begin{verbatim}
Function minPos(A,n)
  If n = 0 Then
    Return 0
  Else
    best <- minPos(A,n-1)
    If A[n-1] < best And A[n-1] > 0 Then
      best <- A[n-1]
    EndIf
    Return max
  EndIf
EndFunction
    \end{verbatim}

    %%%%%%%%%%%%%%
    %%%% TODO %%%%
    %%%%%%%%%%%%%%
    % Give your recursive invariant and your proof here.

\end{document}
